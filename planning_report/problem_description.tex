\section{Problem Description}


The general description of the localization problem is that the car
jointly estimate its pose and the positions of landmarks relative to
the car, known as the \gls{SLAM} problem. Included in this problem is
to find, identify and map new landmarks based on sensor
measurements. The generic form of the \gls{SLAM} problem can be stated
as
\begin{subequations}
  \begin{align}
    x_{k+1} & = f_k(x_k, u_k,v_k), \\
    m_{k+1} & = m_k, \\
    y_k & = h_k(x_k, m_k, u_k) + e_k.
  \end{align}
\end{subequations}
Here, at the discrete time-points $k$, $x_k$ contains information about
the state of the vehicle and $m_k$ contains the location of the
landmarks. The ensemble of the landmarks, $m_{1:k}$ are also denoted as the map,
and it is assumed that positions of the landmarks remain constant over
time.

The dynamics of the vehicle is embedded in the function $f_k$, and the observations from the sensors are embedded in the function $h_k$. Each sensor type has an observation model associated with it, and multiple sensors can be parametrized differently. 

Where the positions of the landmarks and the state is jointly
estimated given all the past measurements. This requires significant
computational resources for large scale problems. The problem to be
solved in this project is to separate the map making and the on-line
localization. The map making can be viewed as a joint smoothing
problem.


Where we again jointly estimates the trajectories  and the positions
of landmarks  given all measurements . However, we can access future
measurements to estimate the state , i.e. non-causal filtering, and
cannot be performed on-line, When we want to perform localization we
calculate


\subsection{Major problems}
Here we list problems that makes this project challenging. These can
be grouped into two groups: external (environment specific) and
internal (technical aspects).
\begin{itemize}
\item External:
  \begin{itemize}
  \item Different timescales in changes of the environment:
    \begin{itemize}
    \item (hour scale) Lightning
    \item (day scale) Different weather conditions
    \item (month) winter summer
    \item (year) new buildings
    \end{itemize}
  \item The size of area to cover
  \end{itemize}
\item Technical:
  \begin{itemize}
  \item  How to efficiently merge information from multiple
    sensors
  \item  How to efficiently merge information from multiple cars.
  \item  How to update map over time.
  \item How to account with multiple measurements.
  \item  10 m error in GPS position.
  \item  How to account for sensor failure.
  \item  How to include new types of sensors in the future.
  \end{itemize}
\item  Computational aspects:
  \begin{itemize}
  \item   Where to perform calculations: in the car or in the
    cloud?
  \item   Where to store a global map, locally or in the cloud?
 \end{itemize}
\end{itemize}


%%% Local Variables:
%%% mode: latex
%%% TeX-master: "main"
%%% End:
