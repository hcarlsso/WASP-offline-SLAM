\section{Problem Description}


The general description of the localization problem is that the car
jointly estimate its pose and the positions of landmarks relative to
the car, known as the \gls{SLAM} problem. Included in this problem is
to find, identify and map new landmarks based on sensor
measurements. The generic form of the \gls{SLAM} problem can be stated
as
\begin{subequations}\label{eq:slam_problem}
  \begin{align}
    x_{k+1} & = f_k(x_k, u_k,v_k), \\
    m_{k+1} & = m_k, \\
    y_k & = h_k(x_k, m_k, u_k) + e_k.
  \end{align}
\end{subequations}
Here, at the discrete time-points $k$, $x_k$ contains information about
the state of the vehicle and $m_k$ contains the location of the
landmarks. The ensemble of the landmarks, $m_{1:k}$ are also denoted as the map,
and it is assumed that positions of the landmarks remain constant over
time.

The dynamics of the vehicle is embedded in the function $f_k$, and the observations from the sensors are embedded in the function $h_k$. Each sensor type has an observation model associated with it, and multiple sensors can be parametrized differently.

Further, the process noise $v_k$ and the measurement noise $e_k$ also plays an important role. In \eqref{eq:slam_problem}, it is assumed that $e_k$ is additive and $v_k$ not necessarily additive. $u_k$ denotes the control input to the system.

The general solution to \eqref{eq:slam_problem} is calculate the posterior distribution of the states and the landmarks given the observations, i.e.
\begin{equation}\label{eq:posterior}
p(x_{1:T}, m_{1:T} | y_{1:T}),
\end{equation}
Thus, the positions of the landmarks and the states are jointly
estimated given all the measurements. In the context of vehicle \gls{SLAM}, a significant amount of sensors are used and the number of landmarks could be large. Thus, for a large scale problem this would require significant computational resources to be solved on-line. This motivates the separation of the map making and
localization into two phases: off-line map-making and on-line localization. However, it is desirable to exclude a dedicated mapping phase and instead create the maps from normal operation. Thus, from a recorded data set the map-making problem is to calculate
\begin{equation}\label{eq:map_making_problem}
p(x_{k}, m_{k} | y_{1:T}),
\end{equation}
for $k = 1, \cdots, T$. The localization problem, given new measurements $y_{1:T}$, is consequently to calculate
$$
p(x_{T}| m ,  y_{1:T}),
$$
where now $m$ is given by \eqref{eq:map_making_problem}.

\subsection{Sensor set-up}\label{sec:sensor-setup}
The sensors available in the test vehicle are the following:
\begin{itemize}
\item Stereo Cameras (Autoliv)
\item GPS+IMU (Applanix LV V5,
  \url{https://www.applanix.com/downloads/products/specs/POSLV_DS_feb_2017_yw.PD
  })
\item  \gls{LIDAR} (Velodyne HD132E)
\item  Radars
\item  Radio based localization (5G).
\item  Wheel encoders
\end{itemize}
Although all sensors could be used in the \gls{SLAM} problem, not all of them
are necessary to provide a solution.

\subsection{Major problems}

Localization and mapping are challenging problems in a car context. The diversity of environments to different weather conditions makes it difficult to create robust and reliable solutions. Aspects of the problem can
be grouped into two groups: external (environment specific) and
internal (technical aspects).
\begin{itemize}
\item External:
  \begin{itemize}
  \item Different time-scales in changes of the environment:
    \begin{itemize}
    \item (hour scale) Lightning
    \item (day scale) Different weather conditions
    \item (month) winter summer
    \item (year) new buildings
    \end{itemize}
  \item The size of area to cover
  \end{itemize}
\item Technical:
  \begin{itemize}
  \item  How to efficiently merge information from multiple
    sensors
  \item  How to efficiently merge information from multiple cars.
  \item  How to update map over time.
  \item How to account with multiple measurements.
  \item  10 m error in GPS position.
  \item  How to account for sensor failure.
  \item  How to include new types of sensors in the future.
  \end{itemize}
\item  Computational aspects:
  \begin{itemize}
  \item   Where to perform calculations: in the car or in the
    cloud?
  \item   Where to store a global map, locally or in the cloud?
 \end{itemize}
\end{itemize}


\subsection{Minimum Viable Problem}

The \gls{MVP} is the general problem reduced while still considered to be a relevant problem. In the \gls{MVP}, the following following simplifications/assumptions:
\begin{itemize}
\item Evaluation of one data-set
\item Static environment
\item Assume all the sensors are calibrated
\item ``Small dataset'', being able to compute on personal computer
\end{itemize}
This scenario could correspond to:
\begin{enumerate}
\item Drive to work and record a single sensor data-set
\item Perform off-line computation
\item Drive the same road to work and navigate using the map
\end{enumerate}
The minimum set of sensors to consider are:
\begin{itemize}
\item GPS
\item Stereo Cameras
\item IMU
\item Radio receivers
\end{itemize}

The localization part should be able to handle multiple sensor types as well as different combinations of them to increase robustness. The system will be evaluated using datasets from
\url{http://robotcar-dataset.robots.ox.ac.uk/}

%%% Local Variables:
%%% mode: latex
%%% TeX-master: "main"
%%% End:
